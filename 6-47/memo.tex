\documentclass[dvipdfmx]{article}
\usepackage{latexsym}
\usepackage{amsmath}
\usepackage{amsthm}
\usepackage{amsfonts}
\usepackage{amssymb}
%\usepackage{array, booktabs}
\usepackage{tikz}
\usepackage{bm}
%\usepackage{framed}
\usepackage{booktabs}
\usepackage{adjustbox}


\newtheorem{definition}{definition}[section]
\newtheorem{hypothesis}{hypothesis}[section]
\newtheorem{theorem}{theorem}[section]
\newtheorem{prop}{proposition}[section]
\newtheorem{lemma}{lemma}[section]
\newtheorem{example}{example}[section]

%----------------------------------------------------------------------
%my macro
\newcommand{\Veps}{\varepsilon}
\newcommand{\norm}[1]{\left\lVert#1\right\rVert}
%----------------------------------------------------------------------

\begin{document}
\section{Problem 47, part(a)}
The choosen function are $(c)$ and $(d)$ in the problem $6-16$.
In case of using Runge-Kutta method, the error bound of $\max_{x \in [a,b]}\left|y(x)-y_h(x)\right|$ from above can be written as 
$\frac{1}{L}\left(\exp\left((b-a)L\right) - 1\right) \tau(h)$, 
where 
\begin{itemize}
  \item $(a,b) = (0, 4)$,
  \item $L = 1$ is the Lipschitz constant of this function, 
  \item $h$ is the step size for the Euler method,
  \item $\tau(h) = \max_{x_n} \left|\frac{1}{h}\left(y(x_{n+1})-y(x_n)\right) - F(x_n, y(x_n), h; f)\right|$,
  \item $F(x_n, y(x_n), h; f) = \frac{1}{6}\left(V_1 + 2V_2 + 3V_3 + V_4\right)$
        defined by the Runge-Kutta method.
\end{itemize}

We confirmed that the numerical errors are bounded by the theoretical estimations as shown in the tables.\\
We also compared the numerical results with the Runge-Kutta method and the Trapezoidal method obtained in the problem $6$-$17$. In all cases, the values of $\max_{x \in [a,b]}\left|y(x)-y_h(x)\right|$ with Runge-Kutta are smaller than the ones with the Trapezoidal method. 
However, even if we use the Runge-Kutta method, the approximations fail when the step size $h$ is not small enough (for example, $h=.5$).

\newpage
\section{Problem 47, part(b)}
We compare the numerical results with the Runge-Kutta method to the fourth order Taylor method obtained in the problem $6$-$46$. When we choose $h=.5$, the results with the Taylor method does not seem to converge correctly. On the other hand, the Runge-Kutta method works even in the case of $h=.5$.
The theoretical bounds from above by the Richardson extrapolation are added in the rightmost columns on the tables.

\end{document} 
