\documentclass[dvipdfmx]{article}
\usepackage{latexsym}
\usepackage{amsmath}
\usepackage{amsthm}
\usepackage{amsfonts}
\usepackage{amssymb}
%\usepackage{array, booktabs}
\usepackage{tikz}
\usepackage{bm}
%\usepackage{framed}
\usepackage{booktabs}
\usepackage{adjustbox}


\newtheorem{definition}{definition}[section]
\newtheorem{hypothesis}{hypothesis}[section]
\newtheorem{theorem}{theorem}[section]
\newtheorem{prop}{proposition}[section]
\newtheorem{lemma}{lemma}[section]
\newtheorem{example}{example}[section]

%----------------------------------------------------------------------
%my macro
\newcommand{\Veps}{\varepsilon}
\newcommand{\norm}[1]{\left\lVert#1\right\rVert}
%----------------------------------------------------------------------

\begin{document}
Problem 6\\
In this problem, the error bound of $\max_{x \in [a,b]}\left|y(x)-y_h(x)\right|$
from above can be written as 
$\frac{1}{L}\left(\exp\left((b-a)L\right) - 1\right) \frac{h}{2} \norm{y^{\prime\prime}}_{\infty}$ 
by the theorem $6.3$ in the textbook, 
where 
\begin{itemize}
  \item $\norm{y^{\prime\prime}}_{\infty} = \max_{x \in [a,b]}\left|y(x)-y_h(x)\right|$,
  \item $(a,b) = (0, 4)$,
  \item $L = 1$ is the Lipschitz constant of this function, 
  \item $h$ is the step size for the Euler method.
\end{itemize}
Hence, this bound is proportional to $h$ because other parameters and values are constants, which implies that if $h$ halves, then the error bound from above does so.\\
We confirmed that this conjecture is true in the numerical results, in which the right most columns on the tables have the theoretical bounds above.\\
Indeed, the value of $\left|y(x)-y_h(x)\right|$ in the second right column at the bottom on each table becomes one half approximately as $h$ halves ($.87218 \to .43737 \to .219$ as $h: .25 \to .125 \to .0625$).

\end{document} 
