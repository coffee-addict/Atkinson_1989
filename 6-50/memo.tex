\documentclass[dvipdfmx]{article}
\usepackage{latexsym}
\usepackage{amsmath}
\usepackage{amsthm}
\usepackage{amsfonts}
\usepackage{amssymb}
%\usepackage{array, booktabs}
\usepackage{tikz}
\usepackage{bm}
%\usepackage{framed}
\usepackage{booktabs}
\usepackage{adjustbox}


\newtheorem{definition}{definition}[section]
\newtheorem{hypothesis}{hypothesis}[section]
\newtheorem{theorem}{theorem}[section]
\newtheorem{prop}{proposition}[section]
\newtheorem{lemma}{lemma}[section]
\newtheorem{example}{example}[section]

%----------------------------------------------------------------------
%my macro
\newcommand{\Veps}{\varepsilon}
\newcommand{\norm}[1]{\left\lVert#1\right\rVert}
%----------------------------------------------------------------------

\begin{document}
Problem $50$\\
When $\lambda = -1$, the numerical results show that the fourth order classical Runge-Kutta method works for this problem. However, even if we take $h$ smaller, the method does not work in the cases of $\lambda = -10$ and $-50$, as shown in the data tables and graphs. So, it is guessed that the method can be stable if $\lambda = -1$, but unstable in the cases of $\lambda = 10$, and $-50$.

\end{document} 
