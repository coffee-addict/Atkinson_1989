\documentclass[dvipdfmx]{article}
\usepackage{latexsym}
\usepackage{amsmath}
\usepackage{amsthm}
\usepackage{amsfonts}
\usepackage{amssymb}
%\usepackage{array, booktabs}
\usepackage{tikz}
\usepackage{bm}
%\usepackage{framed}
\usepackage{booktabs}
\usepackage{adjustbox}


\newtheorem{definition}{definition}[section]
\newtheorem{hypothesis}{hypothesis}[section]
\newtheorem{theorem}{theorem}[section]
\newtheorem{prop}{proposition}[section]
\newtheorem{lemma}{lemma}[section]
\newtheorem{example}{example}[section]

%----------------------------------------------------------------------
%my macro
\newcommand{\Veps}{\varepsilon}
\newcommand{\norm}[1]{\left\lVert#1\right\rVert}
%----------------------------------------------------------------------

\begin{document}
Problem 17\\
We choose (c) and (d) from the problem $16$.\\
The error bound of $\max_{x \in [a,b]}\left|y(x)-y_h(x)\right|$
from above can be written as 
$\frac{1}{L}\left(\exp\left(2L(b-a)\right) - 1\right) \frac{h^2}{12} \norm{y^{\prime\prime\prime}}_{\infty}$ 
by the estimation $(6.5.18)$ in the textbook,
where 
\begin{itemize}
  \item $\norm{y^{\prime\prime\prime}}_{\infty} = \max_{x \in [a,b]}\left|y^{\prime\prime\prime}(x)\right|$,
  \item $(a,b) = (0, 10)$,
  \item $L = 1$ (for (c) and (d)) is the Lipschitz constant of this function, 
  \item $h$ is the step size for the midpoint and trapezoidal method.
\end{itemize}
We confirm that $\max_{x \in [a,b]}\left|y(x)-y_h(x)\right|$ is certainly bounded by the theoretical extimation in the numerical results.\\
Similarly to the discussion in the problem $6$, this bound is proportional to $h^2$, which implies that if $h$ halves, then the error bound from above becomes the quarter of it, which looks true to some extent in the numerical results.

We can see the affects of the ieterations from the tables.
The more iterations, the more accurate results.

Also, for (d), we figure out that the step sizes $h = .5$ or $.25$ are not enougth small to obtain satisfactory numerical results.
Indeed, in the graphs of $h \ge .25$, we see that the approximations clearly fail as $x$ approaches to the right end.\\
Then we additionally compute the result of $h = .1$ and $.01$.\\
In case of $h = .1$, iterations improve the approximations, but still the results are not close to the true values near $x = 10$.\\
On the other hand, $h = .01$ with iteration $J = 3$ looks satisfactory as seen in the last graph.

\end{document} 
